\section{考察}

本研究で提案した「データハーベスト」は,脆弱性情報の収集・統合・履歴管理を通じて,脆弱性管理のあり方に新たなパラダイムを提示する.本章では,その学術的および実務的な貢献を考察する.

本手法の核心は,脆弱性情報のある時点の情報だけでなく更新履歴までを管理対象とし、それをバージョン管理システムで体系的に管理するフレームワークを確立した点にある.これにより,従来は単一の時間に留まっていた脆弱性管理に対して「時間軸に沿った脆弱性情報の観測と分析」という観点を付加する.

このアプローチがもたらす最大の貢献は,脆弱性管理プロセスにおける説明責任と客観性の向上である.Gitによる履歴管理は,脆弱性情報のいかなる変更(例:ステータス変更や情報の追加・削除)も,共有可能な客観的事実として記録する.これにより,「なぜ検知結果が変わったのか」という問いに対し,従来のような推測ではなく,コミットハッシュと差分に基づいた明確な回答が可能になる.脆弱性データベースをハッシュ値で固定することによる検知結果の再現性保証は,プロセスの属人性を軽減し,過去の状況を正確に説明する上で不可欠な機能となる.

また,extractステージは,多様なデータソースの「方言」を吸収する高度な抽象化レイヤーとして機能し,開発者を各仕様の複雑さから解放する.我々が構築し公開するデータベースは,コミュニティがデータ収集という「車輪の再発明」を避けることを可能にし,オープンソースエコシステム全体の技術水準向上にも貢献する.

以上のことから,本手法は脆弱性情報の動的な性質を捉え,その管理に客観性,再現性,説明責任をもたらすことで,現代の脆弱性管理が直面する課題への有効な解決策となる.これは,日々の運用を効率化する実務的価値と,脆弱性情報を時間軸で管理するという新たな手法による学術的意義を併せ持つと結論づける.
