\section{議論}

本章では,提案手法およびその実装について,実際の運用経験を踏まえた上で,将来の改善点と課題について議論する.

\subsection{データサイズの最適化}
現状のシステムでは,広範な脆弱性情報を継続的に収集し履歴管理を行っているが,ベンダーやフォーマットによっては元の脆弱性情報がギガバイトを超えるサイズになる場合がある.これを履歴管理すると,データサイズはさらに増大する.現在のクラウドベースのシステムやストレージにおいては,ギガバイトオーダーのデータを保持すること自体は難易度が低いものの,脆弱性検知のためにこれらのデータを効率的に配布し,継続的に更新し続けるためには,データサイズをできるだけ小さく保つことが望ましい.

データサイズを削減するための候補となる手段を以下に示す.

\subsubsection{シリアライズフォーマットの改善}

現状はJSONをシリアライズフォーマットとして非構造化データに近い形で保持しているが,よりサイズ効率の高いフォーマットへの移行が考えられる.

\begin{description}
  \item[スキーマ情報を利用したシリアライズ] \mbox{} \\
    例えば,Protocol Buffersのようなスキーマベースのシリアライズ形式を用いることで,JSONよりも小さいサイズでデータを保持することが可能である.これにより,ネットワーク転送量やストレージ使用量を削減できる.
  \item[より構造化データに近い形でのシリアライズ] \mbox{} \\
    カラム型データストア(e.g. Parquet\cite{parquet})にデータを格納することで,大幅な圧縮効果が期待できる.ただし,脆弱性情報には再帰構造が含まれるため,その構造化とアクセス方法について慎重な設計が必要となる.これが実現できれば,カラム型ストアの特性を活かした高い圧縮率と効率的なデータアクセスが期待できる.
\end{description}

\subsubsection{データベース内部での圧縮}
現状,脆弱性データベースはJSONがそのまま格納されたBoltDB形式をZstandardで圧縮して配布し,利用時に展開してストレージに配置している.DB内部に圧縮機能を持ったエンジンを利用することで,配布時と利用時の両方でサイズを削減できる可能性がある.候補の例としては,SQLite3のプラグインで圧縮機能を提供するもの(e.g. sqlite-zstd\cite{sqlite-zstd})や,LSM-treeベースのデータベースにおけるブロック単位の圧縮機能(e.g. RocksDB\cite{rocksdb}, Pebble\cite{pebble})が挙げられる.これらの技術を導入することで,ストレージ効率と配布効率のさらなる向上が期待される.

\subsection{差分の論理的な説明能力の向上}
現在,履歴管理は実現されているものの,時系列差分の確認方法はJSON文字列の簡易な行ベースの差分に限定されている.このような差分には,更新日付のみの変更のような軽微なものから,未修正だった脆弱性があるバージョンで修正されたといった利用者にとって極めて重要な情報まで,様々な種類の変更が混在している.

脆弱性に関して重要な変更点を,人間に分かりやすい形で表現できるようになれば,脆弱性への対応にかかる人的負荷を大幅に軽減できる.そのためには,JSON文字列としてではなく,構造化されたデータとして差分を抽出・解析できる機能が望ましい.これにより,以下のような機能を提供できるようになるだろう.
\begin{description}
  \item[重要履歴の抽出] \mbox{} \\
    ある期間で多くの変更点が入っている場合でも,利用者にとって特に重要な差分がある履歴だけを効率的に抽出する機能.
  \item[論理的な差分提示] \mbox{} \\
    単純な行ベース差分ではなく,脆弱性のステータス変更,影響範囲の更新,修正バージョンの追加など,より論理的な構造として差分を提示する機能.
\end{description}

これらの機能は,利用者が脆弱性情報の変化を迅速に理解し,適切な対応を決定する上で大きな助けとなる.

\subsection{データ生成パイプライン(CI)の安定性}
ベンダの脆弱性情報を取得する部分において,相手サーバーの応答遅延や,全く応答しないといったエラーが発生することがあり,これがデータの更新間隔の長期化を招くことがある.また,相手先サーバー側の仕様変更などによって取得プログラムがエラーとなり,プログラムが修正されるまでしばらく更新が停止することも起こり得る.

本システムのデータ生成パイプラインであるCIのさらなる安定化は重要な課題である.これには,リトライ機構の強化,エラー発生時の自動通知と復旧プロセスの改善,そしてベンダー側のAPI変更に対する柔軟な対応メカニズムの導入などが含まれる.パイプラインの安定性は,提供される脆弱性情報の鮮度と信頼性を直接的に左右するため,継続的な改善が必要である.
